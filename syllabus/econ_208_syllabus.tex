\documentclass[11pt]{article}

\usepackage{fullpage}

%% color for the links
\usepackage[usenames, dvipsnames]{color}
\usepackage{longtable}
\usepackage{graphicx}

% We will generate all images so they have a width \maxwidth. This means
% that they will get their normal width if they fit onto the page, but
% are scaled down if they would overflow the margins.
\makeatletter
\def\maxwidth{\ifdim\Gin@nat@width>\linewidth\linewidth \else\Gin@nat@width\fi}
\makeatother
\let\Oldincludegraphics
\includegraphics

%% hyperlinks
\usepackage[colorlinks=true, urlcolor=MidnightBlue, plainpages=false,]{
    hyperref
}% color for the links
\setlength{\parindent}{0pt}
\setlength{\parskip}{6pt plus 2pt minus 1pt}
\setlength{\emergencystretch}{3em} % prevent overfull lines
\setcounter{secnumdepth}{0}

\title{Programming for economists}
\date{Spring 2026}

\begin{document}
    \maketitle

    \begin{flushleft}
        \section{Contact Information}
        \label{contact-information}

        Instructor: Chase Coleman\\ Office: KFT 452\\ Email: cc257@rice.edu\\
        Office Hours: TBA

        \section{Course Objectives and Learning Outcomes}
        \label{course-objectives-and-learning-outcomes}

        Economics is increasingly becoming a compute-first discipline where individuals
        are expected to know how to work with data, numerical tools, and algorithms.

        This course introduces students to programming with Python in the context
        of economic applications. Students will learn how to clean, manipulate,
        and visualize real-world economic data using libraries such as pandas, numpy,
        and matplotlib. Emphasis is placed on understanding how messy data are handled
        in practice and how programming empowers economists to answer pressing policy
        and research questions.

        \section{Required Texts and Materials}
        \label{required-texts-and-materials}

        No textbooks required. There are relevant and useful materials on the QuantEcon
        website and we will occasionally reference the site but all relevant material
        is online and open source.

        All material will be delivered to students in the form of Jupyter Notebooks.

        \section{Exams and Papers}
        \label{exams-and-papers}

        There will be two exams and a single class project.

        The project objectives will be described in a separate document but we
        have found that an independent (small groups allowed) project where
        students choose their own topic of interest is one of the most rewarding
        ways to learn. It allows students to focus in on certain aspects of the course
        that they find most interesting and explore those more deeply than they otherwise
        would have been able to.

        \section{Grade Policies}
        \label{grade-policies}

        There will be 4 categories that make up your grade:

        \begin{itemize}
            \item \textit{Code practice} (25\%): These will be homework
                assignments and we have found that students who complete their
                homework assignments tend to keep up with the material better
                and earn ``easy'' points for their final grade. I will drop your lowest
                score here. You may work collaboratively with other students on
                the code practices.

            \item \textit{Exam} (30\%): There will be two exams. The exam is usually
                administered as a take-home exam with open note and open internet
                (if this is abused, we will have to change this policy). This should
                be individual work and should not be shared with others.

            \item \textit{Project} (40\%): The semester-long project will be the
                biggest component of your grade. We purposefully give you lots of
                freedom here so that you can pursue something that you are interested
                in.

            \item \textit{Completion} (5\%): We expect students to participate
                in class. If you have questions, please dono't be shy about
                asking them.
        \end{itemize}

        All deadlines are a hard-stop and late work will not be accepted.

        When something is due ``end of day'', you may interpret that to be
        submit the assignment sometime before 02:30 the following day, i.e. if I
        say something is due ``end of day'' Wednesday, then you are permitted to turn
        it in as late as 2:30 am on Thursday morning (though I recommend getting
        in the habit of submitting it earlier than that).

        \section{Absence Policies}
        \label{absence-policies}

        I prefer you attend class and it will be a component of your completion
        grade. If there is a reason that you cannot attend then please let me know
        beforehand.

        \section{Rice Honor Code}
        \label{rice-honor-code}

        In this course, all students will be held to the standards of the Rice Honor
        Code, a code that you pledged to honor when you matriculated at this
        institution. If you are unfamiliar with the details of this code and how it
        is administered, you should consult the Honor System Handbook at
        \url{http://honor.rice.edu/honor-system-handbook/}. This handbook outlines
        the University's expectations for the integrity of your academic work,
        the procedures for resolving alleged violations of those expectations, and
        the rights and responsibilities of students and faculty members throughout
        the process.

        \section{Disability Resource Center}
        \label{disability-resource-center}

        If you have a documented disability or other condition that may affect
        academic performance you should: 1) make sure this documentation is on
        file with the Disability Resource Center (Allen Center, Room 111 /
        \href{mailto:adarice@rice.edu}{adarice@rice.edu} / x5841) to determine the
        accommodations you need; and 2) talk with me to discuss your accommodation
        needs.

        \section{Syllabus Change Policy}
        \label{syllabus-change-policy}

        This syllabus is only a guide for the course and is subject to change
        with advanced notice.

        \section{Course Schedule}
        \label{course-schedule}

        Recognize that this is a tentative schedule. Depending on class
        interests and questions, we may spend more time on certain topics and less
        time on others.

        \subsection{2026-01-16}
        \label{section}

        Syllabus + Course expectations + Python introduction

        \subsection{2026-01-23}
        \label{section-1}

        Continue the Python introduction

        \subsection{2026-01-30}
        \label{section-2}

        Organizing data with pandas

        \subsection{2026-02-06}
        \label{section-3}

        Cleaning, reshaping, and merging with pandas

        \subsection{2026-02-13}
        \label{no-class-1}

        No class

        \subsection{2026-02-20}
        \label{section-4}

        numpy + scipy

        \subsection{2026-02-27}
        \label{section-5}

        Visualization tools in Python

        \subsection{2026-03-06}
        \label{section-6}

        Split-apply-combine with pandas + First exam

        \subsection{2026-03-13}
        \label{section-7}

        APIs and labor statistics

        \subsection{2026-03-20}
        \label{no-class-2}

        No class

        \subsection{2026-03-28}
        \label{section-8}

        Introduction to SQL

        \subsection{2026-04-03}
        \label{section-9}

        Polars and duckdb

        \subsection{2026-04-10}
        \label{section-10}

        An introduction to Markov chains + a ``Lake model'' of employment

        \subsection{2026-04-17}
        \label{section-11}

        Applying the Lake model to data

        \subsection{2026-04-24}
        \label{section-12}

        Making maps + Project presentations
    \end{flushleft}
\end{document}